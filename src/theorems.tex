\usepackage{amsthm}

\theoremstyle{plain}
\newtheorem{theorem}{Teorema}[section]
\newtheorem{lemma}{Lema}[section]
\newtheorem{proposition}{Proposición}[section]
\newtheorem{corollary}{Corolario}[theorem]

\theoremstyle{definition}
\newtheorem{definition}{Definición}[section]
\newtheorem{example}{Ejemplo}[section]
\newtheorem{method}{Método}
\newtheorem*{editorial}{Nota Editorial}

\theoremstyle{remark}
\newtheorem{remark}{Observación}[section]

\crefformat{theorem}{#2teorema~#1#3}
\Crefformat{theorem}{#2Teorema~#1#3}
\crefformat{lemma}{#2lema~#1#3}
\Crefformat{lemma}{#2Lema~#1#3}
\crefformat{proposition}{#2proposición~#1#3}
\Crefformat{proposition}{#2Proposición~#1#3}
\crefformat{corollary}{#2corolario~#1#3}
\Crefformat{corollary}{#2Corolario~#1#3}

\crefformat{definition}{#2definición~#1#3}
\Crefformat{definition}{#2Definición~#1#3}
\crefformat{example}{#2lema~#1#3}
\Crefformat{example}{#2Lema~#1#3}
\crefformat{method}{#2método~#1#3}
\Crefformat{method}{#2Método~#1#3}

\crefformat{remark}{#2observación~#1#3}
\Crefformat{remark}{#2Observación~#1#3}

% % \newenvironment{method}[1]{%
% \par\noindent \textbf{#1}. %
% }{}

