
\section{Resolución numérica de problemas de valor inicial para EDO}

\begin{equation}
(PVI)
\begin{cases}
    y'(t)=f(t.y(t)) \\
    y(a) = y_0 \\
    t\in[a,b]
\end{cases}
\end{equation}
\begin{itemize}
        \item Los problemas que trataremos serán de orden 1 siempre.
        \item Se resolverán discretizando el intervalo: $a=t_0<t_1<\dots < t_n\geq b$. Se aproxima $y(t_i)\approx w_i$, buscando un error aceptable en todo punto. Si queremos el valor exacto en puntos intermedios usamos una interpolación adecuada.
\end{itemize}

\section{Métodos de paso fijo. Método de Euler}

\begin{definition} Decimos que un método es de paso fijo si $t_i-t_{i-1}=h,\forall i=1,\dots n$ para cierto $h>0$ constante, siendo $n=\lceil \frac{b-a}{2} \rceil$. \end{definition}

\begin{definition} 
El método de Euler de paso fijo viene dado por:

\begin{equation}
\begin{cases}
    w_0=y_0 \\
    w_{i+1}=w_i + h\cdot f(t_i, w_i)
\end{cases}
\end{equation}

Para $t_i =t_0+i\cdot h,\forall i=0,\dots, n=\lceil \frac{b-a}{2} \rceil$ para cierto $h>0$.

\end{definition}

\begin{theorem}[Convergencia del método de Euler]
    Sea $f:D\subset \mathbb{R}^2\rightarrow \mathbb{R}$ con $D$ abierto, e $Y(t)$ una solución de $(1)$ con $(t,Y(t))\in D,\forall t\in[a,b]$. Sea también la solución numérica $(2)$ para cierto $h$ fijo. Si se cumple
    \begin{enumerate}[label=(\alph*)]
        \item f Lipschitziana respecto de la segunda variable en $D$, con constante de Lipschitz $K$.
        \item $Y''$ existe en todo $[a,b]$ y está acotada por una constante $C\geq 0$.
        \item 
    \end{enumerate}
    Entonces $$\max_{0\leq i \leq n}|Y(t_i)-y_h(t_i)| \leq e^{(b-a)K}\cdot |Y(a)-y_0| + \frac{e^{(b-a)K}-1}{2K}ch$$

\end{theorem}
\begin{proof}

\end{proof}
